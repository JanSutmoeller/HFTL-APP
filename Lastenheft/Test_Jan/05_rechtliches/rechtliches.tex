\section{Rechtliche Aspekte\cite{ct}}

Im privaten Umgang mit \ac{UAV} sollte sich die Frage gestellt werden: Was ist rechtlich erlaubt und was muss beachtet werden? Die Nutzung von Fluggeräten jeglicher Art ist im \ac{LuftVG}\cite{luftvg} und der \ac{LuftVO}\cite{luftvo} geregelt. Dazu werden im nächsten Abschnitt einige wichtige Punkte zusammengefasst.

\subsection{Genehmigungspflicht und generelle Flugverbote}
Für den rein privaten Gebrauch wird keine Aufstiegsgenehmigung vorausgesetzt, sofern das \ac{UAV} nicht über 5 kg Startgewicht hat. Wenn der Einsatz eines \ac{UAV} einen kommerziellen, gewerblichen Grund hat, bedarf es einer Genehmigung seitens des Gesetzgebers. Flugplätze sind von Überflügen ausgeschlossen. In einigen Bundesländern gibt es zudem ausgewiesene Flugverbotszonen
(Regierungsviertel, Atomkraftwerke, Menschenansammlungen und andere).
\subsection{Einschränkungen der örtlichen und personellen Nutzung}
Das \ac{UAV} sollte in Sichtweite des Piloten gesteuert werden(200- 300m). Die Flughöhe sollte, je nach Bundesland, 30 bis 100 Meter nicht überschreiten. Grundstücke dürfen nur mit Erlaubnis überflogen werden. Gerade bei niedrigen Überflügen mit Kameravorrichtung stellt es einen Eingriff in die Privatsphäre dar. Bei der Steuerung von privaten, nicht genehmigungspflichtigen \ac{UAV} gilt keine Altersbeschränkung.
\subsection{Haftung}
Der Führer des \ac{UAV} haftet für Schäden, die durch den Flug entstehen. Haftpflichtversicherungen schließen solche Schäden aus. Es gibt spezielle Versicherungsangebote für den Betrieb von \ac{UAV}. Diese sind auch Voraussetzung für eine Aufstiegsgenehmigung.
\subsection{Nutzung des Bildmaterials}
Ein äußerst wichtiger Punkt ist die Regelung zur Nutzung von Kamerafunktionen. Bildaufnahmen von Personen sind sehr streng geregelt. So ist es in vielen Fällen notwendig sich Genehmigungen von den jeweiligen Personen, Grundstückeigentümern oder Veranstaltern einzuholen. Bilder auf denen Personen explizit zu erkennen sind bedürfen immer das Einverständnis der Personen, da sonst das Recht am eigenen Bild verletzt wird. Ausnahme sind dabei öffentliche Veranstaltungen, wenn die betreffenden Personen nicht das Hauptmotiv des Bildes sind, sondern die Veranstaltung an sich. Motive von Gebäuden dürfen, solange sie die öffentlich zugängliche Seite belichtet, in privatem Umfeld gezeigt werden. Sollten die Bilder veröffentlicht werden, so unterliegen Luftaufnahmen nicht der Panoramafreiheit\cite{Panoramafreiheit}. Das bedeutet es dürfen nur Bilder, die von öffentlichen Orten ohne Zuhilfenahme von Hilfsmitteln (Hubschrauber, \ac{UAV}, Leitern) gemacht werden, genutzt und veröffentlicht werden. Militärisch genutzte Bereiche und Geräte dürfen nicht aufgenommen werden, sobald es zu Nachteilen für die Sicherheit der Bundesrepublik Deutschland kommen kann.
