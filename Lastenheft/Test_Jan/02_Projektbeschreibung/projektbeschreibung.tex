\section{Projektbeschreibung}

\subsection{Einführung und Zielbestimmung}

\subsubsection{Beschreibung des Unternehmens}
Die Hochschule für Telekommunikation Leipzig (HfTL) ist eine private, staatlich anerkannte Fachhochschule. Träger der HfTL ist die HFTL Trägergesellschaft mbH, eine Beteiligungsgesellschaft der Deutschen Telekom AG. Die Schule befindet sich im Leipziger Stadtteil Connewitz. Es werden sowohl Direkt- als auch duale Studiengänge und berufsbegleitende Studiengänge angeboten.

\subsubsection{Beschreibung und Hintergrund der geplanten Softwareentwicklung}
Im Rahmen des Studienmoduls Software-Engineering erarbeiten mehrere Gruppen ein Projekt zur Entwicklung einer neuen Software. Die Leistung wird als Prüfungsleistung anerkannt.
Dieses Projekt wird von der BKMI13 entwickelt. Da es derzeit noch keine Möglichkeit gibt Noten und Stundenpläne auf dem Smartphone komfortabel anzuzeigen, erschien der Nutzen für eine APP mit diesen Funktionen als sehr sinnvoll.
\\
\\
Das kam aus der Vorlage:
Hier geben Sie einen kurzen Überblick über die Gründe und den erwarteten Nutzen der zu entwickelnden Software. Folgende Punkte sollten berücksichtigt werden:
•	Technische und betriebswirtschaftliche Ziele
•	Budgetvorgaben 
•	Kurze Beschreibung der Implementierungsidee (bei Neuentwicklung eines Programms aus einem Entwurf) oder 
•	Kurze Beschreibung der Anpassungen (bei vorhandener Software oder vorhandenen Webanwendungen)
•	Einführung, Betrieb, Wartung 


 
\subsection{Produktübersicht und Einsatz}

\subsubsection{aktuelle Situation}

   \begin{itemize}
      \item Noten
      \begin{itemize}
         \item sind auf QIS hinterlegt
         \item Für Zugriff muss man via Browser auf die Seite zugreifen und sich einloggen
         \item für iPhone Nutzer gibt es eine kostenpflichtige APP (Grades), die Noten aus QIS auslesen kann
      \end{itemize}
      \item Stundenpläne
      \begin{itemize}
      	\item sind auf QIS hinterlegt
      	\item sind ohne Login einsehbar ----Ist das so??????
      	\item man muss sich umständlich zu seinem entsprechenden Studiengang durchklicken
      	\item der Stundenplan kann als iCal heruntergeladen werden
      	\item ebenso sind die Teletutorien hier zu finden
      \end{itemize}
      \item News
      \begin{itemize}
      	\item News stehen gesondert auf der HFTL-Homepage
      	\item https://www.hft-leipzig.de/de/studierende/service/news.html
      	\item kein Login notwendig, öffentlich zugänglich
      \end{itemize}
   \end{itemize}

Beschreiben Sie, wie Sie aktuell arbeiten, und was bisher nicht optimal oder in Ih-rem Sinne funktioniert. 
•	Beschreibung des Ist-Zustands: 
o	Wie wird bisher gearbeitet?
o	Mit welchen Tools oder Systemen?
o	Welche Problembereiche und Schwachstellen gibt es?

\subsubsection{Beschreibung des SOLL-Konzepts}

•	Hier nur Kurzbeschreibungen der Erweiterungen oder Plugins in Listen-form
•	Auch Prozessbeschreibungen

\subsubsection{Beschreibung von Schnittstellen und Techniken}

Sollen vorhandene Daten in die neue Software übernommen werden? 
•	Übernahme von Altdaten 


\subsubsection{Abkürzungen, Nomenklatur, fachliche Zusammenhänge, Datenhierarchie}
????????????????????

\subsection{Produktdetails}

\subsubsection{funktionale Anforderungen}

Zeigen Sie alle Funktionen auf, die Sie von den Erweiterungen erwarten. Beson-ders hilfreich sind Skizzen oder Mockups, welche die Funktionalität visualisie-ren. Gehen Sie außerdem ein auf: 
•	Benutzeroberfläche
•	Logik
•	Rechte
•	Mehrsprachigkeit

\subsubsection{nichtfunktionale Anforderungen(Leistungen, Daten}

Neben Angaben zur Leistungsfähigkeit der Erweiterungen fließen hier auch be-reits vertragliche Erwägungen zwischen Ihnen und dem Entwickler mit ein, falls dieser die Erweiterung auch warten soll. Gehen Sie hier auf die Nutzerzahl ein, welche die Erweiterung gleichzeitig bewältigen soll, sowie die voraussichtlichen Datenmengen. Soll die Erweiterung leicht änderbar sein oder das Prinzip leicht übertragbar? Wie einfach sollen Installation und Wartung sein? Sollen diese vom Auftragnehmer oder vom Auftraggeber durchgeführt werden (können)?
•	Nutzerzahl
•	Antwortzeiten
•	Konformitäten, Änderbarkeit, Übertragbarkeit
•	Installation, Wartbarkeit 


\subsection{Qualitätsanforderungen}

Gehen Sie hier auf mögliche Qualitätsstandards ein, welche die Software zum Bei-spiel im Rahmen einer Zertifizierung oder interner Richtlinien erfüllen soll. 
•	Anforderungen an den Anbieter 
o	Leistungsfähigkeit
o	Erfahrung
o	Unternehmensgröße (Größere Unternehmen haben möglicherweise mehr Ressourcen, kleine Unternehmen sind flexibler und haben kür-zere Kommunikationswege.)
o	Zertifizierung
•	Risikoakzeptanz (Haben Sie die Möglichkeit, Neues auszuprobieren, oder müssen Sie bestimmte Standards einhalten und Richtlinien erfüllen?) 
•	Gesetzesvorgaben
•	interne Richtlinien

\subsection{Betrieb}

Wie und wo soll die Software betrieben werden? Wer soll sie warten und wie schnell soll der Support reagieren können? Soll eine Webanwendung intern oder von einem Dienstleister gehostet werden? Welche sonstigen Supportleistungen erwarten Sie? In Bezug auf die Verfügbarkeit und die Wartung geben Sie hier Min-destverfügbarkeiten und geforderte Antwortzeiten vom Support an. 
•	interner / externer Betrieb
•	Wartungsleistungen
•	Supportleistungen
•	Verfügbarkeit
•	Reaktionszeit des Supports 

\subsection{Projektorganisation}

Wie soll die Zusammenarbeit zwischen Ihnen und dem Entwickler gestaltet sein? 
•	Mitwirkungsleistungen des Kunden, Abgrenzung der Verantwortlichkeiten
•	Test- und Abnahmekonzepte
•	Lieferumfang
•	Anforderungen an die Dokumentation

\subsection{zeitliche Vorgaben und Deadlines}

Zuletzt sollten Sie den zeitlichen Ablauf des Projektes bestimmen. 
•	Wann soll das Projekt starten?
•	Wann soll die Testphase beginnen? 
•	Wann soll das System in die produktive Nutzung übernommen werden?

\subsection{Ergänzungen}

weqwe
