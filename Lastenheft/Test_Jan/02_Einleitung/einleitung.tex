\section{Einleitung}

Der Begriff der Drohne begegnet uns in Deutschland überall im Alltag: In den Medien, Bekannte kaufen sich fast vollkommen autark fliegende Flugmaschinen. An Drohnen montierte, hochauflösende Kameras schießen Bilder, wie sie sonst nur von Fotografen aus Propellermaschinen gemacht werden konnten oder von Satelliten mit Millionen teurem Equipment \cite{holzapfel}. Weltweit herrschen große Diskussionen. Befürworter wie auch Kritiker stehen sich mit nachvollziehbaren Argumenten gegenüber. Es zieht sich ein schmaler Grat zwischen Nutzen und Problemen durch die Thematik. In Deutschland herrscht eine politisch gesetzliche Debatte über den neu aufkommenden Hype um die \glqq private\grqq\ Drohnen (\ac{UAV})\cite{welchedrohne}. 

In dieser Hausarbeit wird zunächst eine terminologische Begriffsbestimmung gemacht und dann auf die Geschichte eingegangen. Daraufhin wird konkretisierend auf die \glqq private\grqq\ Drohne DJI Phantom 2 Vision Plus eingegangen. Zum Schluss wird die rechtliche Situation in Deutschland in Bezug auf  \glqq private\grqq\ Drohnen erläutert.

