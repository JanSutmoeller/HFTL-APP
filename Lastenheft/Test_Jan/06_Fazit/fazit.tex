\section{Fazit}
Die Nutzung der \ac{UAV} hat sich in den letzten Jahren massiv erhöht. Im militärischen Sektor finden sie schon seit langem Beachtung. Seit das erste Mal per \ac{UAV} getötet wurde (2007) stehen sie ständig in der Kritik. Meiner Meinung nach wird sich die Technik dieser fortschrittlichen Technologie schnell weiterentwickeln. Militärverantwortliche werden dieses vorantreiben, da ein Krieg ohne eigenen personellen Einsatz ein Traum eines jeden Landes ist. Gebremst werden kann diese Entwicklung nur durch die ständige Diskussion über Ethik und Nutzen. Schnell wird das Töten aus einem Lagezentrum, mehrere tausend Kilometer von dem eigentlichen Krisengebiet zum \glqq Computerspiel\grqq\ ohne emotionale Verbundenheit zu den Geschehnissen.\\
Im privaten Sektor steigen die Verkaufszahlen von \ac{UAV} stetig. Ich denke es wird sich dem Thema nochmal in der Gesetzgebung gewidmet und die Regeln für den privaten Gebrauch sollten strenger gehandhabt werden. Ein Ausspähen der Nachbarschaft ist durch die hochauflösenden Kameras kein Problem mehr. Des weiteren sollte immer im Blick behalten werden, dass durch diese Möglichkeiten auch ein Missbrauch nicht ausgeschlossen werden kann. Es sollten Gesetze zum Schutze der Zivilbevölkerung aufgestellt werden, die es verhindern eine lückenlose, unbemerkte Überwachung aus der Luft zu etablieren. Ein Missbrauch dieser Technologie hätte schwierige Konsequenzen für die Freiheit des Menschen.\\
Ob es in naher Zukunft Post- und Paketzustellung vorwiegend nur noch per \ac{UAV} geben wird ist fraglich, da meiner Meinung nach ein so hohes Aufkommen an autark fliegenden Objekten zusammen mit den von Menschenhand gesteuerten Flugzeugen ein zu hohes Sicherheitsrisiko darstellen. Sollte es möglich werden die problematischen Sicherheitrisiken zwischen Mensch und Maschine in den Griff zu bekommen, halte ich es für möglich, dass wir ein Zeitalter der Drohnen erleben werden. Bezugnehmend auf den Überwachungsaspekt mache ich mir in Deutschland derzeit wenig Sorgen. Meiner Meinung nach spielt die Freiheit der Menschen in Deutschland eine große Rolle. Ein Überwachen per \ac{UAV} würde sich schwer durchsetzen.
Nichtsdestotrotz sollten wir gelegentlich in den Himmel schauen und schauen was uns da so beobachtet!\\
